\begin{abstract}
A major goal of nuclear theory is to model atomic nuclei and make theoretical predictions of nuclear observables starting from inter-nucleon forces. Approaches starting from these inter-nucleon forces, known as ab-initio methods, face significant computational challenges due to the complexity of the nuclear system and the nature of the forces. The similarity renormalization group (SRG) method is often used in modern calculations to soften these interactions, which simplifies the problem thereby allowing ab-initio methods to be extended to larger systems. SRG, when applied to an $A$-particle system, induces many-body forces that are not accounted for when the $A$-body results are used to compute results for larger systems. The errors from these omitted induced many-body forces currently limit the application of SRG to small and medium systems, as for large systems these errors become too large for calculations to yield useful results. The SRG method takes an input from the user, the flow operator $\hat{G}_s$, which is taken to be the kinetic energy in most modern SRG calculations. Results have suggested that alternative choices for the flow operator may produce smaller induced many-body forces, which would allow calculations with SRG to be extended to larger systems.

We return to the 1-dimensional system of bosons in which the nuclear theory group at OSU initially explored the application of SRG to nuclear problems. As the results from this simple system are generalizable to full 3-dimensional calculations, we seek to test alternative flow operators in this 1-dimensional system, where visualizing and interpreting results is substantially easier. We develop a Python library to handle the setup of the physical system and the SRG evolution. We compare the results obtained using this library to the results from an analogous paper by Jurgenson and Furnstahl in 2008 to verify the correctness of our implementation. We then use this framework to test induced 3-body forces for several 2-body flow operator choices. We discuss our preliminary results and offer some options for further exploration.
\end{abstract}
