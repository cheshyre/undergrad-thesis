\chapter{Similarity Renormalization Group Formalism}

In this chapter, we introduce the physics concepts and formalism necessary to understand the 1-dimensional problems to which SRG is applied. We the introduce SRG and discuss details and open questions regarding its use.

\section{Quantum Mechanics Operators}

In quantum mechanics, the primary objects are operators and states

\subsection{Jacobi Coordinates}

\subsection{Harmonic Oscillator States with Proper Symmetry}

\section{Similarity Renormalization Group}

The similarity renormalization group (SRG), whose use in nuclear physics was initially explored at OSU, is one method of reducing the computational complexity of low-energy nuclear calculations. The idea behind it is to continuously unitarily transform the operator of interest (for example, the Hamiltonian) into a simpler form. This simpler form is chosen to allow the large basis to be truncated without affecting the operator eigenvalues, which are essential to the truncated operator's utility in later calculations.