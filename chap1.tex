\chapter{Introduction}

general flow: introduction of origin of nuclear theory -> structure of nucleus and nuclear matter -> problems of nuclear theory (how nuclei work, astrophysical applications (origin of elements), nucleus as a laboratory in the search for BSM physics, applications in national security, energy, medicine) -> explanation of past struggles with computational difficulty (phenom models) -> explanation that these models have low predictive power outside the domain in which they were fit -> need for better nuclear theory calculations in light of experimental advances in low-energy nuclear physics (FRIB) and astrophysics (LIGO results suggesting site of r-process) -> rapid expansion of range of calculations from first principles in recent years due to improvements in computational hardware and development of new computational methods -> focus on modern ab-initio methods (NCSM, IM-SRG, lattice QCD), RG methods, EFT methods -> use of these methods is state of the art in low-energy nuclear theory -> study of their correct, optimal application to calculations is an open problem


STILL WIP

\begin{itemize}
    \item{How general to start?}
    \item{How can I communicate the general problems of nuclear theory to a layperson?}
    \item{Introduction to Hamiltonian formalism?}
    \item{Will emphasize computational complexity due to basis size}
\end{itemize}


\section{Similarity Renormalization Group}

The similarity renormalization group (SRG), whose use in nuclear physics was initially explored at OSU, is one method of reducing the computational complexity of low-energy nuclear calculations. The idea behind it is to continuously unitarily transform the operator of interest (for example, the Hamiltonian) into a simpler form. This simpler form is chosen to allow the large basis to be truncated without affecting the operator eigenvalues, which are essential to the truncated operator's utility in later calculations.

\subsection{Jacobi Coordinates}

\subsection{Harmonic Oscillator States with Proper Symmetry}
