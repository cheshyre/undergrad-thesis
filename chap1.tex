\chapter{Introduction}

STILL WIP

\begin{itemize}
    \item{How general to start?}
    \item{How can I communicate the general problems of nuclear theory to a layperson?}
    \item{Introduction to Hamiltonian formalism?}
    \item{Will emphasize computational complexity due to basis size}
\end{itemize}


\section{Similarity Renormalization Group}

The similarity renormalization group (SRG), whose use in nuclear physics was initially explored at OSU, is one method of reducing the computational complexity of low-energy nuclear calculations. The idea behind it is to continuously unitarily transform the operator of interest (for example, the Hamiltonian) into a simpler form. This simpler form is chosen to allow the large basis to be truncated without affecting the operator eigenvalues, which are essential to the truncated operator's utility in later calculations.

\subsection{Jacobi Coordinates}

\subsection{Harmonic Oscillator States with Proper Symmetry}
